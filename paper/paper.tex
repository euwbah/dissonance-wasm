\documentclass[12pt]{homework}
\usepackage[utf8]{inputenc}
\newcommand{\hwname}{}
\newcommand{\hwemail}{euwbah@gmail.com}
\newcommand{\hwtype}{}
\newcommand{\hwnum}{}
\newcommand{\hwclass}{Polyadic Dissonance and Tonicity Measure}
\newcommand{\hwlecture}{Course Notes}
\newcommand{\hwsection}{}
\usepackage{mymacros2}
\allowdisplaybreaks

\theoremstyle{definition}
\newtheorem{definition}{Definition}
\newtheorem{example}{Example}
\newtheorem{theorem}{Theorem}
\newtheorem{corollary}{Corollary}
\newtheorem{lemma}{Lemma}
\theoremstyle{plain}

\begin{document}
\maketitle

\setcounter{questionCounter}{0}
\setcounter{section}{1}
\question*{Intro to ...}

\def\T{\mathbf{T}}
\def\R{\mathbf{R}}
\def\TR{\mathbf{TR}}

In the context of \(CEG\) where \(\T(C) = \T(C \mid CEG)\).

For shorthand, write \(\T(C \mid CE) \R(CE) = \TR_{CE}(C)\).

\begin{align*}
    \R(CEG) &= \T(C \mid CE)
\end{align*}

\end{document}
